\documentclass{tufte-handout}

%\geometry{showframe}% for debugging purposes -- displays the margins

\usepackage{amsmath}

% Set up the images/graphics package
\usepackage{graphicx}
\setkeys{Gin}{width=\linewidth,totalheight=\textheight,keepaspectratio}
\graphicspath{{graphics/}}

\title{Protein and Amino Acid Catabolism}
\author{}
\date{}  % if the \date{} command is left out, the current date will be used

% The following package makes prettier tables.  We're all about the bling!
\usepackage{booktabs}

% The units package provides nice, non-stacked fractions and better spacing
% for units.
\usepackage{units}

% The fancyvrb package lets us customize the formatting of verbatim
% environments.  We use a slightly smaller font.
\usepackage{fancyvrb}
\fvset{fontsize=\normalsize}

% Small sections of multiple columns
\usepackage{multicol}

% Provides paragraphs of dummy text
\usepackage{lipsum}

% These commands are used to pretty-print LaTeX commands
\newcommand{\doccmd}[1]{\texttt{\textbackslash#1}}% command name -- adds backslash automatically
\newcommand{\docopt}[1]{\ensuremath{\langle}\textrm{\textit{#1}}\ensuremath{\rangle}}% optional command argument
\newcommand{\docarg}[1]{\textrm{\textit{#1}}}% (required) command argument
\newenvironment{docspec}{\begin{quote}\noindent}{\end{quote}}% command specification environment
\newcommand{\docenv}[1]{\textsf{#1}}% environment name
\newcommand{\docpkg}[1]{\texttt{#1}}% package name
\newcommand{\doccls}[1]{\texttt{#1}}% document class name
\newcommand{\docclsopt}[1]{\texttt{#1}}% document class option name

\begin{document}

\maketitle% this prints the handout title, author, and date

\begin{abstract}
\noindent This lecture will discuss protein breakdown, and then further catabolism of amino acids into energy or other products.  Amino acids are broken down into both carbon skeletons (which often feed into the glycolysis or the TCA cycle) and their amino groups (often transfered to Glutamate/Glutamine for storage, or released via the Urea Cycle).  This unit will discuss the function and the regulation of these processes.  For more details on amino acid breakdown and the Urea cycle please refer to Chapter 30 in Biochemistry: A Short Course, available in reserve\cite{Berg2015}.
\end{abstract}

\tableofcontents

\pagebreak
\section{Learning Objectives}

\begin{itemize}
\item Stuff
\end{itemize}

\section{Protein Breakdown}

\subsection{Mechanisms of Protein Degradation}

\subsection{Endocrine and Metabolic Signals of Protein Breakdown}

\section{Amino Acid Catabolism}

\subsection{Branched-Chain Amino Acid Catabolism}

\newthought{The nitrogen groups from amino acids are often transfered to glutamate}

\subsection{Regulation of Glutamate Dehydrogenase}

\subsection{The Urea Cycle}



\newthought{The activity of the Urea Cycle is controlled by the levels of carbamoyl phosphate.}

\newthought{The primary location for the urea cycle is the liver}
\bibliography{library}
\bibliographystyle{plainnat}

\end{document}
