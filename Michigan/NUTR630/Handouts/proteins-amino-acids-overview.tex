\documentclass{tufte-handout}

%\geometry{showframe}% for debugging purposes -- displays the margins

\usepackage{amsmath}

% Set up the images/graphics package
\usepackage{graphicx}
\setkeys{Gin}{width=\linewidth,totalheight=\textheight,keepaspectratio}
\graphicspath{{graphics/}}

\title{Protein and Amino Acid Synthesis}
\author{}
\date{}  % if the \date{} command is left out, the current date will be used

% The following package makes prettier tables.  We're all about the bling!
\usepackage{booktabs}

% The units package provides nice, non-stacked fractions and better spacing
% for units.
\usepackage{units}

% The fancyvrb package lets us customize the formatting of verbatim
% environments.  We use a slightly smaller font.
\usepackage{fancyvrb}
\fvset{fontsize=\normalsize}

% Small sections of multiple columns
\usepackage{multicol}

% Provides paragraphs of dummy text
\usepackage{lipsum}

% These commands are used to pretty-print LaTeX commands
\newcommand{\doccmd}[1]{\texttt{\textbackslash#1}}% command name -- adds backslash automatically
\newcommand{\docopt}[1]{\ensuremath{\langle}\textrm{\textit{#1}}\ensuremath{\rangle}}% optional command argument
\newcommand{\docarg}[1]{\textrm{\textit{#1}}}% (required) command argument
\newenvironment{docspec}{\begin{quote}\noindent}{\end{quote}}% command specification environment
\newcommand{\docenv}[1]{\textsf{#1}}% environment name
\newcommand{\docpkg}[1]{\texttt{#1}}% package name
\newcommand{\doccls}[1]{\texttt{#1}}% document class name
\newcommand{\docclsopt}[1]{\texttt{#1}}% document class option name

\begin{document}

\maketitle% this prints the handout title, author, and date

\begin{abstract}
\noindent This lecture will cover mechanisms and signals of both protein synthesis and amino acid biosynthesis, for the non-essential amino acids.  Protein building is important for growth as well as tissue repair.  This lecture will also cover in more detail why some amino acids are essential or conditionally essential in our diet.
\end{abstract}

\tableofcontents

\pagebreak
\section{Learning Objectives}

\begin{itemize}
\item Understand the mechanistic differences that explain the difference between dispensable and indispensable amino acids.
\item Evaluate the roles of insulin, growth hormone, testosterone and cortisol on protein synthesis and degradation.
\item Describe the central role of glutamate as a pool of nitrogen.
\item Describe the relationships between the glycolytic and TCA cycle intermediates and amino acid biosynthesis.
\item Explain why cysteine and tyrosine are dispensable only if precursors are available.
\item Understand how amino acid biosynthetic rates are controlled by utilization and by negative feedback.
\item Understand the role that the indispensable amino acids play in controlling protein synthesis.
\end{itemize}

\section{Protein Synthesis is a Tigthly Regulated Process}

\subsection{The Rate of Protein Synthesis Depends on the Levels of Available Amino Acids}

\subsection{Several Endocrine Signals Regulate Protein Biosynthesis}

\newthought{Several of these hormonal signaling systems converge on mTORC1.}

\subsection{Protein Synthesis is Energetically Expensive}

\section{Synthesis of and Control of Amino Acids}

\subsection{Humans Have Lost the Ability to Synthesize Several Amino Acids}

\newthought{A good example is Glytamine synthesis from Glutamate.}

\subsection{Several Amino Acids are Synthethized from Essential Amino Acids}

\subsection{Non-Essential Amino Acids Are Derived from Glycolytic and TCA Cycle Intermediates.}

\subsection{Glutamate is a Key to Maintaining Nitrogen Atoms for Amino Acid Synthesis}

Glutamate is a key part of several \emph{transaminase} reactions.  These are near-equillibrium reactions where an amino group is transfered fom glutamate to another amino acid, or vice versa.  Some examples are below:

 \begin{equation}\label{eq:alt}
\alpha KG + Ala \rightleftharpoons Glu + Pyr
\end{equation}

 \begin{equation}
\alpha KG + Asp \rightleftharpoons Glu + OAA
\end{equation}

 \begin{equation}
\alpha KG + Val \rightleftharpoons Glu +\alpha Ketoisovalerate
\end{equation}

Since these are easily reversible reactions, the directionality depends on the concentrations of products and substrates on each side.  For example in equation \ref{eq:alt}, if there is high levels of Glutamate and Pyruvate, then  Alanine and $\alpha$-ketoglutarate will be produced.  Because Glutamate and $\alpha$-ketoglutarate are present on both sides of most transaminase reactions, this is one way in which TCA cycle intermediates ($\alpha$-ketoglutarate) and amino acids (\textit{i.e.} Glutamate) are kept in balance.

\newthought{Glutamate is a non-toxic carrier of nitrogen.}  During amino acid breakdown\sidenote{This will be covered in the next lecture}, several amino acids can be converted to glutamate via transaminases, then glutamate releases its amino group via the functions of Glutamate Dehydrogenase:

\begin{equation}
Glu + H_2O + NAD^+  \rightarrow  \alpha KG + NH_3 + NADH + H^+
\end{equation}

In humans this is irreversible, as we cannot re-synthesize glutamate from ammonia.  The ammonia released from this reaction is released into the Urea cycle\sidenote{Also covered in the next lecture}.

\section{Protein Requirements and Nitrogen Balance}
As we just mentioned, when amino acids are being used, ammonia is generated. This can be measured by urinary nutrogen levels.  If dietary nitrogen and urinary nitrogen are equal, then a person is said to be in \emph{Nitrogen Balance}.  During periods of protein catabolism, urinary nitrogen is higher than intake.  During periods of protein synthesis, urinary nitrogen is lower\emph{This is because the dietary nitrogen containing amino acids are not being oxidized.}.  This is one way by which dietary requirements are determined, since the lack of any essential amino acid causes proteins to be degraded to release the essential amino acids.  Now there will be an excess of the non-limiting amino acid, which will then be oxidized and released as urea.

\bibliography{library}
\bibliographystyle{plainnat}

\end{document}
