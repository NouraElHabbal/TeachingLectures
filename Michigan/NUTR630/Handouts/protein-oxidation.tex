\documentclass{tufte-handout}

%\geometry{showframe}% for debugging purposes -- displays the margins

\usepackage{amsmath}

% Set up the images/graphics package
\usepackage{graphicx}
\setkeys{Gin}{width=\linewidth,totalheight=\textheight,keepaspectratio}
\graphicspath{{graphics/}}

\title{Protein and Amino Acid Catabolism}
\author{}
\date{}  % if the \date{} command is left out, the current date will be used

% The following package makes prettier tables.  We're all about the bling!
\usepackage{booktabs}

% The units package provides nice, non-stacked fractions and better spacing
% for units.
\usepackage{units}

% The fancyvrb package lets us customize the formatting of verbatim
% environments.  We use a slightly smaller font.
\usepackage{fancyvrb}
\fvset{fontsize=\normalsize}

% Small sections of multiple columns
\usepackage{multicol}

% Provides paragraphs of dummy text
\usepackage{lipsum}

% These commands are used to pretty-print LaTeX commands
\newcommand{\doccmd}[1]{\texttt{\textbackslash#1}}% command name -- adds backslash automatically
\newcommand{\docopt}[1]{\ensuremath{\langle}\textrm{\textit{#1}}\ensuremath{\rangle}}% optional command argument
\newcommand{\docarg}[1]{\textrm{\textit{#1}}}% (required) command argument
\newenvironment{docspec}{\begin{quote}\noindent}{\end{quote}}% command specification environment
\newcommand{\docenv}[1]{\textsf{#1}}% environment name
\newcommand{\docpkg}[1]{\texttt{#1}}% package name
\newcommand{\doccls}[1]{\texttt{#1}}% document class name
\newcommand{\docclsopt}[1]{\texttt{#1}}% document class option name

\begin{document}

\maketitle% this prints the handout title, author, and date

\begin{abstract}
\noindent This lecture will discuss protein breakdown, and then further catabolism of amino acids into energy or other products.  Amino acids are broken down into both carbon skeletons (which often feed into glycolysis or the TCA cycle) and their amino groups (typically transfered to Glutamate/Glutamine for storage, or released via the Urea Cycle).  This unit will discuss the functions and the regulation of these processes.  For more details on amino acid breakdown and the Urea cycle refer to Chapter 30 in Biochemistry: A Short Course, available in reserve\cite{Berg2015}.
\end{abstract}

\tableofcontents

\pagebreak
\section{Learning Objectives}

\begin{itemize}
\item Evaluate the signals that lead to protein degradation and how those signals activate proteolysis and autophagy.
\item Explain the circumstances by which amino acids and proteins would be degraded.
\item Describe the fates of the carbon skeletons when amino acids are catabolized.
\item Consider the importance of branched chain amino acids, and describe the regulation of branched-chain ketoacid dehydrogenase.
\item Explain the key role, and the mechanisms of regulation of Glutamate Dehydrogenase in amino acid catabolism.
\item Describe the role of the urea cycle and analyse how defects in the urea cycle could be detected and treated.
\end{itemize}

\section{Protein Breakdown}

As we discussed in the last unit, there is no easily accessible depot for amino acids, like how glycogen functions for glucose storage.  Essential amino acids can obly be obtained via the diet or by breaking down proteins, and breaking down proteins comes at a cost because most proteins play key functional roles.  Furthermore, making proteins is energetically very costly, so breaking down proteins should only be done when absolutely necessary.  As such, protein breakdown into amino acids is very tightly controlled in cells.

\subsection{Mechanisms of Protein Degradation}

There are two main mechanisms by which proteins can be broken down, through the proteasome or through the lysosome.  The proteasome is a large multi-subunit complex \citep{Finley2009}.  Individual proteins are targetted, usually by the addition of ubiquitin\sidenote{itself a small protein}.  Once targetted, these individual proteins move to the proteasome and there are degraded into individual amino acids.  This can be done at a protein-specific level, so for example a protein that the cell wants to get rid of, can be very specifically targetted and removed.  The specificity of this targetting is via a specific enzyme called an E3 Ubiquitin Ligase.  These proteins recognize a specific protein, and target it for ubiquitinylation.  An example of this in muscle tissue is an E3 ligase called MuRF1\sidenote{Muscle ring finger 1}.  During muscle atrophy, MuRF1 activity is increased, which targets myofibrillar proteins for ubiquitinylation and degradation \citep{Bodine2014}.  

\newthought{The other main protein degrading process is called autophagy.}  In autophagy\, rather than targetting individual proteins, entire organelles (like a mitochondria) can be engulfed and are broken down within lysosomes.  The lipids (via lipases) and proteins (via proteases) are broken down within the lysosomes\sidenote{membrane enclosed organelles containing a variety of enzymes typically for degrading macromolecules, for more information see \url{https://www.ncbi.nlm.nih.gov/books/NBK9953/}}.  In relation to the proteasome this is much less specific, but has a much higher capacity.  Autophagy\sidenote{Which led to the 2016 Nobel Prize in Medicine and Physiology, see \url{https://www.nobelprize.org/nobel_prizes/medicine/laureates/2016/press.html}} is often upregulated during times of amino acid starvation.  As such, mTORC1 activity\sidenote{which is decreased during amino acid or energy deprivation} is a potent \emph{inhibitor} of autophagy \citep{Noda1998}.

\newthought{Extracellular proteins such as collagen are broken down via the secretion of proteases.}  To digest proteins outside the cell a variety of enzymes known as matrix metalloproteases are secreted by cells.  These degrade parts of the extracellular matrix including collagen, elastin and fibronectin.

\subsection{Endocrine and Metabolic Signals of Protein Breakdown}

One protein degradation signal that we have discussed previously is cortisol.  This glucocorticoid signals to muscle cells to break down proteins into their constituent amino acids, largely to provide gluconeogenic substrates to the liver.  The primary route of action of glucocorticoids is thought to be the FOXO-dependent transcriptional activation of the atrogenes MuRF1 and Atrogin.  These ubiquitin ligases then target muscle proteins for degradation and amino acid release.  This is one mechanism by which chronic stress, or prescription glucocorticoids\sidenote{such as prednisone, corticosterone or dexamethasone} result in muscle weakness.

Another factor in the regulation of proteolysis is the \emph{reduction in anabolic signals}.  Insulin and mTORC1 are both potent suppressors of proteasome and autophagosome function, so reductions in these signaling pathways often promotes protein breakdown.  This is thought to be especially important when the mTORC1 activators Leucine, Arginine and Lysine are depleted.  Another consideration is that during insulin resistance\sidenote{For example in obese pre-diabetic or diabetic individuals.}, insulin signaling in the muscle is reduced and protein breakdown can be accelerated \citep{Wang2006b}.  This can increase gluconeogenesis\sidenote{By providing more substrates.} and reduce exercise capacity.

\section{Amino Acid Catabolism}

Proteins are broken down for two major reasons:

\begin{itemize}
\item To free up essential amino acids for other protein synthetic requirements.
\item To provide energy or generate glucose via amino acid oxidation or gluconeogenesis respectively.
\end{itemize}	

In the former example, generally cellular/organismal energy requirements are met and protein synthesis wants to occur, but amino acids are needed.  In the latter example, the cell or organism requires energy or glucose, but protein synthesis is largely inactive.  The removal of an amino group\sidenote{A process called deamination.} from an amino acid is typicaly irreversible.  This is an especially important consideration for essential amino acids, since these now must be provided by the diet.  As you might expect, the regulation and disposal of these amino groups is extremely tightly regulated.  We will discuss the regulation of these processes first, then discuss the fates of the remainder of the amino acid\sidenote{Which we refer to as the carbon skeleton.}.  Remember, for amino acids to be oxidized from proteins two decisions must be made, first the protein must be broken down into amino acids, and then those amino acids must be deaminated and/or oxidized.


\newthought{The nitrogen groups from amino acids are often transfered to glutamate.}  As we discussed previously, there are a series of transaminase reactions.  These are used for both biosynthetic and degradation purposes.  These enzymes are particularly important for removing the amino group from an amino acid, leaving a carbon skeleton (also sometimes refered to as an $\alpha$-ketoacid) and glutamate.  We have discussed ALT and AST previously, but in Table \ref{tab:transaminases} a longer list of mammalian transaminases.

% Please add the following required packages to your document preamble:
% \usepackage{booktabs}
\begin{table}[b]
\centering
\caption{Mammalian transaminases}
\label{tab:transaminases}
\begin{tabular}{@{}rllrl@{}}
\toprule
\multicolumn{2}{c}{\textbf{Substrates}} & \textbf{Enzyme} & \multicolumn{2}{c}{\textbf{Products}}   \\ \midrule
Alanine          & $\alpha$-Ketoglutarate      & ALT             & Pyruvate                    & Glutamate \\
Aspartate        & $\alpha$-Ketoglutarate      & AST             & Oxaloacetate                & Glutamate \\
Leucine          & $\alpha$-Ketoglutarate      & BCAT            & $\alpha$-Ketoisocaproate           & Glutamate \\
Valine           & $\alpha$-Ketoglutarate      & BCAT            & $\alpha$-Ketoisovalerate           & Glutamate \\
Isoleucine       & $\alpha$-Ketoglutarate      & BCAT            & $\alpha$-Ketomethylvalerate        & Glutamate \\
Tyrosine         & $\alpha$-Ketoglutarate      & TAT             & 4-hydroxyphenylpyruvate     & Glutamate \\
Tryptophan       & $\alpha$-Ketoglutarate      & TTA             & (indol-3-yl)pyruvate        & Glutamate \\
Methionine       & $\alpha$-Ketoglutarate      & MAT             & 2-oxo-4-methylthiobutanoate & Glutamate \\
Cysteine         & $\alpha$-Ketoglutarate      & CT              & Sulfinylpyruvate            & Glutamate \\ \bottomrule
\end{tabular}
\end{table}

A couple of points to highlight here.  The non-amino acid substrate in all of these cases is the TCA cycle intermediate $\alpha$-Ketoglutarate, which makes this step of amino acid degradation highly cataplerotic, since in the absence of $\alpha$-Ketoglutarate, there is no receptor for the amino group.  This also means that amino acid breakdown is driven by a buildup of the free amino acids in the presence of available TCA cycle intermediates.  Second, since all transaminase reactions are rapid, equillibrium reactions, the concentration of glutamate is important for this first step.  If glutamate is building up in the cell, then these amino acids will not be degraded to the same extent.

\newthought{You may notice that several amino acids are not shown here.}  As we have discussed previously Glutamine is converted to Glutamate via the Glutaminase enzyme, while Asparagine is processed to Aspartate using a similar enzyme, Asparaginase.  Phenylalanine is converted first to Tyrosine\sidenote{Via Phenylalanine Hydroxylase, the enzyme that is deficient in individuals with PKU.}.  Arginine, Histidine, Lysine and Proline have complex paths but end up as Glutamate as well.  The only two exceptions to nitrogen flow through Glutamate are Threonine and Serine.  These are both catabolized into Glycine, which is broken down in to CO$_2$ and ammonia via the Glycine Cleavage System\sidenote{A series of enzymes that completely catabolize glycine in several steps.}.

\subsection{Regulation of Glutamate Dehydrogenase}

The flow of nitrogen during amino acid catabolism is generally amino acid to Glutamate then Glutamate to ammonia.  This second step is controlled by Glutamate Dehydrogenase (GDH), which catalyzes the \emph{irreversible} reaction shown in Reaction \ref{eq:GDH}.  This primary location of this enzyme is mitochondrial, so for amino acids to be oxidized they (or the Glutamate derived from them) must be transported into the mitochondria.

\begin{equation}\label{eq:GDH}
Glu + H_2O + NAD^+  \rightarrow  \alpha KG + NH_4^+ + NADH + H^+
\end{equation}

This reaction replenishes the $\alpha$-Ketoglutarate consumed in the transaminase step, while also releasing Glutamate's amino agroup as ammonia ($NH_4^+$) and while generating a molecule of NADH\sidenote{Typically worth 2.5 ATP equivalents}.  As you may be able to guess, for an irreversible enzyme of such importance, GDH is under multiple sets of allosteric control.  Before you read any further, take a minute to think under which conditions you think that this reaction would proceed.

If your thoughts were \emph{energy needs} and \emph{amino acid surplus} you are on the right track!  The main positive regulators of GDH are ADP, GDP, NAD$^+$ and branched chain amino acids (especially Leucine).  The main inhibitors include GTP, NADH and Palmitoyl-CoA\sidenote{This is the first step in the degradation of the fatty acid palmitate, and indicates that there are sufficient fatty acids to use as fuel, rather than breaking down proteins}.  In general that means that Glutamate will be irreversibly broken down only when energy needs are high, essential amino acids are high and fatty acids are low.  More details on the allosteric regulation of GDH can be found in a review article by \citet{Smith2008}.  

At a post-translational level, GDH is also under the control of mTORC1 signaling.  mTORC1\sidenote{Which as you may recall is activated by some amino acids, insulin and high energy levels.} is thought to work through an enzyme called SIRT4.  SIRT4 adds an ADP-ribose group to GDH, thus inactivating this enzyme.  Therefore when mTORC1 is active, GDH becomes active \citep{Csibi2013}.  One hypothesis is that this could be a pathway by which nutrient excess leads to the oxidation of un-needed amino acids, especially Glutamine and Glutamate.

\subsection{The Urea Cycle}

The final nitrogen products of Glutamate Dehydrogenase, Glutaminase, Arginase and the Glycine Cleavage System\sidenote{Which ends the catabolism of Serine, Threonine and Glycine.} is ammonia.  Ammonia is very toxic to cells and organs, a condition known as hyperammonia.  As such, ammonia needs to be efficiently converted to the less damaging, and more easily excreted molecule Urea.  The urea cycle begins in the mitochondria, typically in the liver via the synthesis of free ammonia and bicarbonate to form a molecule called Carbamoyl Phosphate (CPS), the first committed step of this cycle.  Somewhat similarly to the TCA cycle, CPS is attached to Ornithine, which goes through several conversion steps, releasing Urea and regenerating Ornithine.  This step, diagramed in reaction \ref{eq:CPS1}, is the key site of regulation of the Urea cycle.

\begin{equation}\label{eq:CPS1}
 NH_4^+ + CO_2 + 2 ATP \rightarrow  CPS
\end{equation}

\newthought{The activity of the Urea Cycle is controlled by the levels of N-Acetylglutamate.}  The activity of the Urea cycle should be coupled to the amounts of amino acid breakdown products, namely Glutamate.  As such, a Glutamate-derived molecule called N-Acetylglutamate (NAG) is a potent allosteric activator of Carbamoyl Phosphate Synthetase, the enzyme which catalyzes reaction \ref{eq:CPS1}.  NAG synthase is itself regulated by Arginine, so that when Amino Acids\sidenote{Specifically Arginine and Glutamate} and Acetyl-CoA\sidenote{Generated either via fatty acid $\beta$-oxidation, Ketogenic Amino Acid catabolism or from Pyruvate Dehydrogenase.} are elevated, NAG increases, which in turn activates the Urea cycle\sidenote{It might help to sketch this out in the margins.}.  Urea cycle enzymes are also upregulated by gluconeogenic hormones including glucagon and cortisol.  This is to ensure that the deamination products of amino acid conversions to glucose are able to be removed.

\subsection{Branched-Chain Amino Acid Catabolism}

The branched-chain amino acids\sidenote{Leucine, Isoleucine and Valine} are under a another level of metabolic control.  Because they are high demand, low availability essential amino acids, it is especially important that their levels remain protected.  The first step in BCAA catabolism is their transamination (see the Branched-Chain Aminotransferase, BCAT in Table \ref{tab:transaminases}).  This reversible reaction equillibrates the pool of BCAA's with their respective $\alpha$-ketoacids\sidenote{Think about what could cause a build-up of these $\alpha$-ketoacids.}.  The rate-limiting step of their catabolism is the next step, mediated by an enzyme called Branched Chain Ketoacid Dehydrogenase (BCKDH).  This enzyme catalyzes the following \emph{irreversible} steps of BCAA catabolism, starting with the transamination products of Leucine, Valine and Isoleucine respectively:

\begin{equation}\label{eq:leu}
\alpha Ketoisocaproate + CoA + NAD^+ \rightarrow IsovalerylCoA + NADH + CO_2
\end{equation}

\begin{equation}\label{eq:val}
\alpha Ketoisovalerate + CoA + NAD^+ \rightarrow IsobutyrylCoA + NADH + CO_2
\end{equation}

\begin{equation}\label{eq:iso}
\alpha Ketomethylvalerate + CoA + NAD^+ \rightarrow \alpha MethylbutyrylCoA + NADH + CO_2
\end{equation}

These reactions are all fairly similar in that they take an $\alpha$ ketoacid and generated an activated form\sidenote{The CoA version; the activation of a product prior to complete oxidation will come up again when we discuss lipid oxidation.}, releasing CO$_2$ and producing NADH).  These activated products then catabolized further as described in the next section.  Since BCKDH is the rate limiting step for all three of these reactions, and is the main control point by which BCAA's are released, it is unsurprising that it is controlled by both internal and external signals.

\newthought{BCKDH is inhibited by protein phosphorylation.}  Similar to PFK2 and Pyruvate Kinase, BCKDH is \emph{inactivated} by protein phosphorylation.  The kinase that is responsible for BCKDH phosphorylation is \emph{inactivated} by a build-up of the branched chain ketoacids, especially the Leucine catabolite $\alpha$ Ketoisocaproate (see reaction \ref{eq:leu}).  In this way, a buildup of the ketoacids turns off the inhibitory protein kinase and allows for BCAA catabolism.  It has been reported that BCKDH expression is induced by glucocorticoids, and reduced by insulin, suggesting that chronic gluconeogenic signals can modify the activity of this process. 

\subsection{The Fates of Amino Acid Derived Carbon Skeletons}

We have focused on the amino groups of the amino acids and how they often end up in the Urea cycle, but what about the rest of the amino acid carbon skeleton?  Processes vary, but generally in steps that are thought to be in near-equillibrium ,the carbon skeletons are catabolized into molecules you are probably already familiar with.  These endpoints are summarized in Table \ref{tab:carbon-skeletons}

\newthought{The ketogenic amino acids, Lysine and Leucine are converted into Acetyl-CoA}, while the partially ketogenic amino acids\sidenote{Phenylalanine, Isoleucine, Threonine and Tyrosine} are broken down into both Acetyl-CoA and another potentially glucogenic molecule.  This is an important difference, because it means that the products of Leucine and Lysine catabolism \emph{can not} enter gluconeogenesis.  That is because Acetyl-CoA cannot become glucose.  Rather it can be catabolized in the TCA cycle, enter \textit{de novo} lipogenesis, or be released as a ketone body.

\begin{table}[]
\centering
\caption{Carbon skeleton fates.  These often involve other side products being generated, but note that most of the amino acids end up as Acetyl-CoA, TCA cycle intermediates (like Fumarate, Succinyl-CoA and $\alpha$ Ketoglutarate or Pyruvate.)}
\label{tab:carbon-skeletons}
\begin{tabular}{lll}
\textbf{Amino Acids}                                        & \textbf{Carbon Skeleton Fate}        & \textbf{Notes}                 \\ \hline
Leucine, Lysine                                    & Acetyl-CoA                  & Ketogenic  \\ \hline
Tyrosine and\\ Phenylalanine                         & Acetyl-CoA and\\ Fumarate     & Partially Ketogenic   \\ \hline
Isoleucine                                         & Acetyl-CoA, Succinyl-CoA & Partially Ketogenic   \\ \hline
Threonine and\\ Tryptophan                           & Acetyl-CoA, Pyruvate     & Partially Ketogenic   \\ \hline
Asparagine and\\ Aspartate                           & Oxaloacetate                & via AST               \\ \hline
Arginine, Proline,\\ Histidine,\\ Glutamine, Glutamate & $\alpha$ Ketoglutarate      & via GDH               \\ \hline
Methionine                                         & Succinyl-CoA                &                       \\ \hline
Cysteine, Alanine                                  & Pyruvate                    &                      
\end{tabular}
\end{table}

\newthought{The gluconeogenic amino acids} on the other hand are either anaplerotic\sidenote{Tyrosine, Phenylalanine, Asparagine, Aspartate, Arginine, Proline, Histidine, Glutamine,  Glutamate and Methionine}, which can either function in the TCA cycle or be converted via PEPCK into phoshpoenolpyruvate, or become Pyruvate\sidenote{Alanine, Threonine, Tryptophan, Cystine}, which can then undergo gluconeogenesis via the activities of Pyruvate Carboxylase and PEPCK.  To think about whether amino acids will be catabolized to energy, or enter gluconeogenesis, consider the metabolites and hormones that govern the rates of gluconeogeneic flux and oxidative phosphorylation.

\bibliography{library}
\bibliographystyle{plainnat}

\end{document}
