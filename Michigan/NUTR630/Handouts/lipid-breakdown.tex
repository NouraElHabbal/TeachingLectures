\documentclass{tufte-handout}

%\geometry{showframe}% for debugging purposes -- displays the margins

\usepackage{amsmath}

% Set up the images/graphics package
\usepackage{graphicx}
\setkeys{Gin}{width=\linewidth,totalheight=\textheight,keepaspectratio}
\graphicspath{{graphics/}}

\title{Regulation of Lipid Breakdown}
\author{}
\date{}  % if the \date{} command is left out, the current date will be used

% The following package makes prettier tables.  We're all about the bling!
\usepackage{booktabs}

% The units package provides nice, non-stacked fractions and better spacing
% for units.
\usepackage{units}

% The fancyvrb package lets us customize the formatting of verbatim
% environments.  We use a slightly smaller font.
\usepackage{fancyvrb}
\fvset{fontsize=\normalsize}

% Small sections of multiple columns
\usepackage{multicol}

% Provides paragraphs of dummy text
\usepackage{lipsum}

% These commands are used to pretty-print LaTeX commands
\newcommand{\doccmd}[1]{\texttt{\textbackslash#1}}% command name -- adds backslash automatically
\newcommand{\docopt}[1]{\ensuremath{\langle}\textrm{\textit{#1}}\ensuremath{\rangle}}% optional command argument
\newcommand{\docarg}[1]{\textrm{\textit{#1}}}% (required) command argument
\newenvironment{docspec}{\begin{quote}\noindent}{\end{quote}}% command specification environment
\newcommand{\docenv}[1]{\textsf{#1}}% environment name
\newcommand{\docpkg}[1]{\texttt{#1}}% package name
\newcommand{\doccls}[1]{\texttt{#1}}% document class name
\newcommand{\docclsopt}[1]{\texttt{#1}}% document class option name

\begin{document}

\maketitle% this prints the handout title, author, and date

\begin{abstract}
\noindent This unit will cover the synthesis of lipids including cholesterol, fatty acid and triglyceride synthesis.  For more details on these topics, refer to Chapters 28 and 29 in Biochemistry: A Short Course available in reserve\cite{Berg2015}.
\end{abstract}

\tableofcontents

\pagebreak
\section{Learning Objectives}

\begin{itemize}
\item Identify the functions of cholesterol within our body
\item Understand how fatty acids are synthesized in the body
\item Describe the initial highly regulated step of FA synthesis
\item Understand the reactions resulting in triacylglycerol synthesis resulting in 3 fatty acids esterified to glycerol
\item Describe the breakdown of triacylglycerol to glycerol plus fatty acids and the fates of these products
\item Describe fatty acid breakdown (beta-oxidation)
\item Determine the amount of energy produced by fatty acid breakdown in comparison to glucose
\item Understand when ketogenesis occurs, what organs ketones are a fuel source for and consequences of overactive ketogenesis
\item Describe how triglyceride synthesis is regulated in the liver




\end{itemize}

\section{Synthesis of Triglycerides from Fatty Acids}

\section{Cholesterol Synthesis}

\subsection{HMG-CoA Reductase is the Rate Limiting Step for Cholesterol Synthesis}

\subsection{Sensing and Regulation of Sterol Biosynthesis by SREBP2}

\section{\textit{De Novo} Fatty Acid Synthesis}

\subsection{Fatty Acid Synthesis from Glucose and Ketogenic Amino Acids}

\subsection{Desaturation of Fatty Acids}

\subsection{Regulation of Fatty Acid Synthesis}

\bibliography{library}
\bibliographystyle{plainnat}

\end{document}
