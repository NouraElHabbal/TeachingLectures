\documentclass{tufte-handout}

%\geometry{showframe}% for debugging purposes -- displays the margins

\usepackage{amsmath}

% Set up the images/graphics package
\usepackage{graphicx}
\setkeys{Gin}{width=\linewidth,totalheight=\textheight,keepaspectratio}
\graphicspath{{graphics/}}

\title{Regulation of Lipid Catbolism}
\author{}
\date{}  % if the \date{} command is left out, the current date will be used

% The following package makes prettier tables.  We're all about the bling!
\usepackage{booktabs}

% The units package provides nice, non-stacked fractions and better spacing
% for units.
\usepackage{units}

% The fancyvrb package lets us customize the formatting of verbatim
% environments.  We use a slightly smaller font.
\usepackage{fancyvrb}
\fvset{fontsize=\normalsize}

% Small sections of multiple columns
\usepackage{multicol}

% Provides paragraphs of dummy text
\usepackage{lipsum}

% These commands are used to pretty-print LaTeX commands
\newcommand{\doccmd}[1]{\texttt{\textbackslash#1}}% command name -- adds backslash automatically
\newcommand{\docopt}[1]{\ensuremath{\langle}\textrm{\textit{#1}}\ensuremath{\rangle}}% optional command argument
\newcommand{\docarg}[1]{\textrm{\textit{#1}}}% (required) command argument
\newenvironment{docspec}{\begin{quote}\noindent}{\end{quote}}% command specification environment
\newcommand{\docenv}[1]{\textsf{#1}}% environment name
\newcommand{\docpkg}[1]{\texttt{#1}}% package name
\newcommand{\doccls}[1]{\texttt{#1}}% document class name
\newcommand{\docclsopt}[1]{\texttt{#1}}% document class option name

\begin{document}

\maketitle% this prints the handout title, author, and date

\begin{abstract}
\noindent  Biochemistry: A Short Course available in reserve\cite{Berg2015}.
\end{abstract}

\tableofcontents

\pagebreak
\section{Learning Objectives}

\begin{itemize}
\item Explain how triglyceride breakdown into glycerol and free fatty acids is controlled in adipocytes by hormonal signals.
\item Explain how high carbohydrate diets affect fuel utilization, including effects on lipid fuel utilization.  Describe at an endocrine level how this is thought to occur.
\item Determine how much energy, in ATP equivalents, is released during the oxidation of a given fatty acid.  Be able to relate the energy content of a fatty acid, in general to its physical properties (length and saturation).
\item Explain the rate limiting steps of lipid oxidation.
\item Explain how ketone bodies are converted to ATP in non-hepatic tissues, and what governs this specificity.
\item Demonstrate an understanding of how how \textit{de novo} lipogenesis and $\beta$-oxidation are reciprocately controlled.
\item Describe how very long chain fatty acids are oxidized differently from long chain fatty acids.
\item Explain how odd-numbered fatty acids are catabolized, including the importance of vitamin B12 in this process. 
\item Evaluate the role of transcriptional regulation and long term adaptations to fatty acid oxidative capacity.
\end{itemize}


\section{Lipolysis Liberates Fatty Acids from Triglyerides}
\subsection{Carbohydrate Overfeeding and Lipid Utilization}
\section{Fatty Acid Oxidation in the Mitochondria}
\subsection{The Role of CPTI in Lipid Oxidation}
\subsection{Determining the Energy Content of a Fatty Acid}
\subsection{Alternative Fatty Acid Catabolism}
\subsection{Ketolysis}
\section{Regulation of Fatty Acid Oxidation}
\subsection{Energy Dependent Regulation via AMPK}
\subsection{Transcriptional Adaptations for Fatty Acid Oxidation}
\newthought{In terms of athletic performance,} increasing the ability to oxidize fatty acids is important for endurance athletes.




\bibliography{library}
\bibliographystyle{plainnat}

\end{document}
