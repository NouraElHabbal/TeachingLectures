\documentclass{tufte-handout}

%\geometry{showframe}% for debugging purposes -- displays the margins

\usepackage{amsmath}

% Set up the images/graphics package
\usepackage{graphicx}
\setkeys{Gin}{width=\linewidth,totalheight=\textheight,keepaspectratio}
\graphicspath{{graphics/}}

\title{Endocrine Control of Growth}
\author{Dave Bridges, Ph.D.}
%\date{24 January 2009}  % if the \date{} command is left out, the current date will be used

% The following package makes prettier tables.  We're all about the bling!
\usepackage{booktabs}

% The units package provides nice, non-stacked fractions and better spacing
% for units.
\usepackage{units}

% The fancyvrb package lets us customize the formatting of verbatim
% environments.  We use a slightly smaller font.
\usepackage{fancyvrb}
\fvset{fontsize=\normalsize}

% Small sections of multiple columns
\usepackage{multicol}

% Provides paragraphs of dummy text
\usepackage{lipsum}

% These commands are used to pretty-print LaTeX commands
\newcommand{\doccmd}[1]{\texttt{\textbackslash#1}}% command name -- adds backslash automatically
\newcommand{\docopt}[1]{\ensuremath{\langle}\textrm{\textit{#1}}\ensuremath{\rangle}}% optional command argument
\newcommand{\docarg}[1]{\textrm{\textit{#1}}}% (required) command argument
\newenvironment{docspec}{\begin{quote}\noindent}{\end{quote}}% command specification environment
\newcommand{\docenv}[1]{\textsf{#1}}% environment name
\newcommand{\docpkg}[1]{\texttt{#1}}% package name
\newcommand{\doccls}[1]{\texttt{#1}}% document class name
\newcommand{\docclsopt}[1]{\texttt{#1}}% document class option name

\begin{document}

\maketitle% this prints the handout title, author, and date

\begin{abstract}
\noindent This lecture covers endocrine control of growth.  The primary hormone that mediates growth is, unsurprisingly known as growth hormone\sidenote{sometimes refered to as somatorop(h)in, hGH, or when generated recombinantly rhGH}.  This lecture covers the following pages in the textbook: 350-353 and 358-359 \cite{Widmaier2013}.
\end{abstract}

\tableofcontents

\pagebreak

\section{Learning Objectives}
For this lecture, the learning objectives are:
\begin{itemize}
\item List the hormones important for growth at key times in a person's life.
\item Describe the functions of human growth hormone on growth (bones and soft tissues), and on metabolism, and the regulation of its secretion.  Explain what 'rhGH' means.
\item State the "dual effector hypothesis" for GH actions, and the relative roles of GH and IGF-1 in growth control. 
\item Describe the interactions among all the key growth-regulating hormones at key times of a person's life: in utero, neonatally, childhood, puberty, adulthood, and senescence.
\item Describe the daily regulation of GH levels and the physiological relevance of these cycles.

\end{itemize}

There are several hormones that are involved in normal growth.  The most important is Growth Hormone, but Insulin, Thyroid Hormones, Vitamin D and sex hormones are also very important.  These are covered in separate lectures.  Generally proper growth (length and mass increase) requires proper nutrition\sidenote{both macro- and micronutrients} and a good psychosocial environment.  

\section{Regulation of Growth Hormone and IGF-1 Levels}

\subsection{Hypothalamic and Pituitary Control of Growth Hormone}

\subsection{Growth Hormone Regulates IGF-1 Secretion}

\subsection{Growth Hormone And Changes During Aging}

\subsection{Diurnal Cycles of Growth Hormone Levels}

\section{Effects of Growth Hormone}

\subsection{Growth Hormone Signaling}

\subsection{Bone and Soft Tissue Growth}

\subsection{Regulation of Metabolism}

\section{Pathologies Associated with Growth Hormone Signaling}

\subsection{Acromegaly}

\subsection{Dwarfism}

\listoffigures
\listoftables

\bibliography{library}
\bibliographystyle{plainnat}



\end{document}
