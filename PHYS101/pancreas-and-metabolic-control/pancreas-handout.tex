\documentclass{tufte-handout}

%\geometry{showframe}% for debugging purposes -- displays the margins

\usepackage{amsmath}

% Set up the images/graphics package
\usepackage{graphicx}
\setkeys{Gin}{width=\linewidth,totalheight=\textheight,keepaspectratio}
\graphicspath{{graphics/}}

\title{Pancreatic Hormones and Metabolic Control}
\author{Dave Bridges, Ph.D.}
%\date{24 January 2009}  % if the \date{} command is left out, the current date will be used

% The following package makes prettier tables.  We're all about the bling!
\usepackage{booktabs}

% The units package provides nice, non-stacked fractions and better spacing
% for units.
\usepackage{units}

% The fancyvrb package lets us customize the formatting of verbatim
% environments.  We use a slightly smaller font.
\usepackage{fancyvrb}
\fvset{fontsize=\normalsize}

% Small sections of multiple columns
\usepackage{multicol}

% Provides paragraphs of dummy text
\usepackage{lipsum}

% These commands are used to pretty-print LaTeX commands
\newcommand{\doccmd}[1]{\texttt{\textbackslash#1}}% command name -- adds backslash automatically
\newcommand{\docopt}[1]{\ensuremath{\langle}\textrm{\textit{#1}}\ensuremath{\rangle}}% optional command argument
\newcommand{\docarg}[1]{\textrm{\textit{#1}}}% (required) command argument
\newenvironment{docspec}{\begin{quote}\noindent}{\end{quote}}% command specification environment
\newcommand{\docenv}[1]{\textsf{#1}}% environment name
\newcommand{\docpkg}[1]{\texttt{#1}}% package name
\newcommand{\doccls}[1]{\texttt{#1}}% document class name
\newcommand{\docclsopt}[1]{\texttt{#1}}% document class option name

\begin{document}

\maketitle% this prints the handout title, author, and date

\begin{abstract}
\noindent This lecture covers endocrine control of appetite.  It covers the following pages in the textbook: 341-344 \cite{Widmaier2013}.  T
\end{abstract}

\tableofcontents

\pagebreak

\section{Learning Objectives}
For this lecture, the learning objectives are:
\begin{itemize}
\item Name three zones in the adrenal cortex and major regulator(s) of each zone.
\item Name three steroidogenesis pathways and their major products.
\item Explain briefly the physiological mechanism of adrenogenital syndrome.
\item Describe the physiological actions and roles of aldosterone.
\item Explain briefly the renin-angiotensin system.
\item Describe the negative feedback regulation of aldosterone and its relationship to blood volume/blood pressure homeostasis.
\item Describe hepatic and extrahepatic metabolic actions of glucocorticoids. Discuss their relationship.
\item State the major findings caused by adrenal hypersecretion of mineralocorticoids.
\item State the major findings caused by adrenal hypersecretion of glucocorticoids. 
\item Name the major hormones secreted from the adrenal medulla. Discuss the differences of epinephrine (epi) and norepinephrine (NE) in cardiovascular actions (physiological levels). 
\item List the major metabolic actions of catecholamines.
\item Contrast the thresholds for actions vs. plasma levels of epi and NE under common conditions, like exercise, and in the disease pheochromocytoma

\end{itemize}



\listoffigures
\listoftables

\bibliography{library}
\bibliographystyle{plainnat}



\end{document}
